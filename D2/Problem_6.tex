\documentclass[12pt,letterpaper]{article}
\usepackage{fullpage}
\usepackage[top=2cm, bottom=4.5cm, left=2.5cm, right=2.5cm]{geometry}
\usepackage{amsmath,amsthm,amsfonts,amssymb,amscd}
\usepackage{lastpage}
\usepackage{enumerate}
\usepackage{fancyhdr}
\usepackage{mathrsfs}
\usepackage{xcolor}
\usepackage{graphicx}
\usepackage{listings}
\usepackage{hyperref}

\hypersetup{%
  colorlinks=true,
  linkcolor=blue,
  linkbordercolor={0 0 1}
}
 
\renewcommand\lstlistingname{Algorithm}
\renewcommand\lstlistlistingname{Algorithms}
\def\lstlistingautorefname{Alg.}

\lstdefinestyle{Python}{
    language        = Python,
    frame           = lines, 
    basicstyle      = \footnotesize,
    keywordstyle    = \color{blue},
    stringstyle     = \color{green},
    commentstyle    = \color{red}\ttfamily
}

\setlength{\parindent}{0.0in}
\setlength{\parskip}{0.05in}

% Edit these as appropriate
\newcommand\course{SOEN 6011}
\newcommand\hwnumber{1}                  % <-- homework number
\newcommand\NetIDa{Robin Karlose}           % <-- NetID of person #1
\newcommand\NetIDb{40089313}           % <-- NetID of person #2 (Comment this line out for problem sets)

\pagestyle{fancyplain}
\headheight 35pt
\lhead{\NetIDa}
\lhead{\NetIDa\\\NetIDb}                 % <-- Comment this line out for problem sets (make sure you are person #1)
\chead{\textbf{\Large Problem 6}}
\rhead{\course \\29th July 2019}
\lfoot{}
\cfoot{}
\rfoot{\small\thepage}
\headsep 1.5em

\begin{document}

\section*{Test Cases}

\textbf{Test Case 1}

Function --- Power.powr(double, int)

Input--- Power.powr(5.0, 3)

Expected --- 125.0

Result --- Pass

The above test case is tied to \textbf{FR3} \newline

\textbf{Test Case 2}

Function --- Power.nat\_log(int)

Input--- Power.powr(20.0)

Expected --- 2.995

Result --- Pass

The above test case is tied to \textbf{FR4} \newline

\textbf{Test Case 3}

Function --- Power.powr(double, double, double)

Input--- Power.powr(16.0,0.75,0.001)

Expected --- 8.0

Result --- Pass

The above test case is tied to \textbf{FR3} \newline

\textbf{Test Case 4}

Function --- Power.epowrx(double)

Input--- Power.epowrx(3.0)

Expected --- 20.085

Result --- Pass

The above test case is tied to \textbf{FR3} \newline


\textbf{Test Case 5}

Function --- BetaFuncStirling.CompBetaS(double , double)

Input--- BetaFuncStirling.CompBetaS(3.0,4.0)

Expected --- 0.016

Result --- Pass , with a very low delta

The above test case is tied to \textbf{FA1, FR1 , FR2}  \newline

\textbf{Test Case 6}

Function --- BetaFuncLnGamma.lngamma(double)

Input--- BetaFuncLnGamma.lngamma(3.5)

Expected --- 1.2009

Result --- Pass , super accurate

The above test case is tied to \textbf{FA1} \newline

\textbf{Test Case 7}

Function --- BetaFuncLnGamma.beta(double,double)

Input--- BetaFuncLnGamma.beta(3.0,4.0)

Expected --- 0.0168

Result --- Pass , super accurate 

The above test case is tied to \textbf{FA1 , FR1 , FR2} \newline

\textbf{Test Case 8}

Function --- BetaFuncLnGamma.beta(double,double)

Input--- BetaFuncLnGamma.beta(-3.0,-4.0)

Expected --- -\infty

Result --- Fail as expected, because it violates FA1 and FA2 , but commented out in the JUnit code

The above test case is tied to \textbf{FA1,FA2}




\end{document}
