\documentclass[12pt,letterpaper]{article}
\usepackage{fullpage}
\usepackage[top=2cm, bottom=4.5cm, left=2.5cm, right=2.5cm]{geometry}
\usepackage{amsmath,amsthm,amsfonts,amssymb,amscd}
\usepackage{lastpage}
\usepackage{enumerate}
\usepackage{fancyhdr}
\usepackage{mathrsfs}
\usepackage{xcolor}
\usepackage{graphicx}
\usepackage{listings}
\usepackage{hyperref}

\hypersetup{%
  colorlinks=true,
  linkcolor=blue,
  linkbordercolor={0 0 1}
}
 
\renewcommand\lstlistingname{Algorithm}
\renewcommand\lstlistlistingname{Algorithms}
\def\lstlistingautorefname{Alg.}

\lstdefinestyle{C}{
    language        = C,
    frame           = lines, 
    basicstyle      = \footnotesize,
    keywordstyle    = \color{blue},
    stringstyle     = \color{green},
    commentstyle    = \color{red}\ttfamily
}

\setlength{\parindent}{0.0in}
\setlength{\parskip}{0.05in}

% Edit these as appropriate
\newcommand\course{SOEN 6011}
\newcommand\hwnumber{1}                  % <-- homework number
\newcommand\NetIDa{Robin Karlose}           % <-- NetID of person #1
\newcommand\NetIDb{40089313}           % <-- NetID of person #2 (Comment this line out for problem sets)

\pagestyle{fancyplain}
\headheight 35pt
\lhead{\NetIDa}
\lhead{\NetIDa\\\NetIDb}                 % <-- Comment this line out for problem sets (make sure you are person #1)
\chead{\textbf{\Large Problem 7}}
\rhead{\course \\2nd August 2019}
\lfoot{}
\cfoot{}
\rfoot{\small\thepage}
\headsep 1.5em

\begin{document}

\section*{Testing and Test Review of F10 - $\sigma$ }

I decided to test the code with the following situations

\begin{itemize}
\item{My own test cases , based on my knowledge of the Standard Deviation Function}
\item{Existing Test cases in the calculatorTest.java file}
\item{Replicated all test cases for both categories of Standard Deviation - Sample and Population}
\end{itemize}


The testing environment was my preferred IDE for Java - IntelliJ.

I executed the program and was presented with a GUI where I manually entered my own test cases.

Also I ran the calculatorTest.java file to see if all test cases presented by the author passed.


\section*{Testing Observations}
% Rest of the work...
The following were my observations during the testing process:- 

\begin{itemize}
\item{The test cases were comprehensive, extremely well written and covered all possible scenarios}
\item{The GUI provided an appropriate error message when I tried to enter anything other than numbers - like characters , strings , special characters etc}
\item{For cases without commas, it was considered one whole number and the error message was different - "Please enter numbers separated by commas"}
\item{Negative numbers were also adequately dealt with}
\item{I did not even notice a small difference in SD when I calculated the SD with the positive equivalents of the same numbers . For instance I got the exact same result when I calculated the SD for "-200,-4.5,-172.89" and the SD for "200,4.5,172.89" }
\item{The calculation for SD was extremely accurate,correct to many decimal places}
\item{Equal support and attention given to both categories of SD - sample and population}
\item{Overall - good robust testing with testing principles and all variety of test cases kept in mind}
\item{Curious to see how this function would perform in integration testing along with other functions of the calculator}
\end{itemize}




\end{document}
