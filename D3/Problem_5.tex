\documentclass[12pt,letterpaper]{article}
\usepackage{fullpage}
\usepackage[top=2cm, bottom=4.5cm, left=2.5cm, right=2.5cm]{geometry}
\usepackage{amsmath,amsthm,amsfonts,amssymb,amscd}
\usepackage{lastpage}
\usepackage{enumerate}
\usepackage{fancyhdr}
\usepackage{mathrsfs}
\usepackage{xcolor}
\usepackage{graphicx}
\usepackage{listings}
\usepackage{hyperref}

\hypersetup{%
  colorlinks=true,
  linkcolor=blue,
  linkbordercolor={0 0 1}
}
 
\renewcommand\lstlistingname{Algorithm}
\renewcommand\lstlistlistingname{Algorithms}
\def\lstlistingautorefname{Alg.}

\lstdefinestyle{C}{
    language        = C,
    frame           = lines, 
    basicstyle      = \footnotesize,
    keywordstyle    = \color{blue},
    stringstyle     = \color{green},
    commentstyle    = \color{red}\ttfamily
}

\setlength{\parindent}{0.0in}
\setlength{\parskip}{0.05in}

% Edit these as appropriate
\newcommand\course{SOEN 6011}
\newcommand\hwnumber{1}                  % <-- homework number
\newcommand\NetIDa{Robin Karlose}           % <-- NetID of person #1
\newcommand\NetIDb{40089313}           % <-- NetID of person #2 (Comment this line out for problem sets)

\pagestyle{fancyplain}
\headheight 35pt
\lhead{\NetIDa}
\lhead{\NetIDa\\\NetIDb}                 % <-- Comment this line out for problem sets (make sure you are person #1)
\chead{\textbf{\Large Problem 5}}
\rhead{\course \\2nd August 2019}
\lfoot{}
\cfoot{}
\rfoot{\small\thepage}
\headsep 1.5em

\begin{document}

\section*{Manual Code Review of F9 - $x^y$ }

I decided to go ahead with a manual code review of my team mates function -   $x^y$ . 


There are a couple of reasons why code review is important,some of which include
\begin{itemize}
\item{Knowledge sharing across members of the team and with members of other teams}
\item{Code legibility}
\item{Consistency of code in large projects}
\item{Correction of accidental errors in code}
\end{itemize}


The approach I took in reviewing the code was I manually evaluated the code in basic details against certain guidelines which I mention in the following points 

\begin{itemize}
\item{\textbf{Purpose} - Whether the code achieve the purpose it was originally designed for and whether it meets all the requirements}
\item{\textbf{Implementation} - How the code was actually implemented and transformed into something which generates output}
\item{\textbf{Legibility and Style} - Whether it followed best practices and certain standard coding styles and whether the code was legible to other users/readers or not }
\item{\textbf{Maintainability} - How maintainable the test code is especially with future functional changes/and or scaling}
\end{itemize}


\section*{Discussion and actual code review}
% Rest of the work...
The following section contains the discussion and actual code review (with some basic details) against the four guidelines discussed in the above 4 points \newline

\textbf{PURPOSE}

\begin{itemize}
\item{The code achieves the authors purpose to some extent by calculating integer powers of numbers , but it cannot calculate $x^y$ for real number values of $y$}
\item{To be fair though , the limitation that it cannot calculate $x^y$ for real number values of $y$ is mentioned in the authors functional requirement documents}
\end{itemize}

\textbf{IMPLEMENTATION}

\begin{itemize}
\item{The code is neatly broken down into 3 classes and 1 test class - the 3 actual classes being APowerB.java , Calculator.java , ICalculator.java  (which is actually an interface) and 1 testing class TestCalculator.java}
\item{There is good use of abstraction especially with the class Calculator.java implementing the interface ICalculator.java}
\item{Code does not make use of any libraries and is fairly dependency free}
\item{If I were to re-implement it , I would go into much more logical detail in the functions and try and implement $x^y$ where $y$ is a real number using some approximation technique}
\end{itemize}


\textbf{LEGIBILITY AND STYLE}

\begin{itemize}
\item{Code is implemented in a simple but extremely lucid manner}
\item{High code readability with presence of comments everywhere}
\item{Code has presence of JavaDoc}
\end{itemize}

\textbf{MAINTAINABILITY}

\begin{itemize}
\item{Presence of a test class called TestCalculator.java}
\item{The function can be used in a scalable way in a calculator}
\item{Proper Integration tests need to be carried out if the function is used in a calculator along with other calculators}
\item{Change Requests (CRs) should not break this existing code}
\item{More detailed comments would help when integrating with other functions of the calculator}
\item{Room for external documentation like ReadMe , User Manual etc}
\end{itemize}


\section*{References}

The following article from www.medium.com greatly helped me and inspired me in my code review.
The link is below:-

\url{https://medium.com/palantir/code-review-best-practices-19e02780015f}



\end{document}
